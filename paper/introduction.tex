\section{Introduction}

Large software systems are notoriously difficult to understand and
debug.  Static analysis tools yield strong proofs about all possible
behaviors, but may be limited by issues of scalability and
decidability.  Furthermore, unsound languages such as C can easily
fail in ways that violate their own semantics; few static analyses are
designed for this contingency.  Testing avoids such issues by directly
running and observing the real code.  However, testing is only as good
as test coverage, and has limited predictive power.  If end users do
things not anticipated by the test suite, they may uncover novel bugs.

Ultimately, the most truthful information about program behavior must
come from the end users themselves.  End users will discover new bugs,
or may collectively prove that a known flaw thought to be rare is
actually a common, recurring problem.  This is especially true in
industrial practice: engineers cost money, delayed releases cost
money, and bug fixing is a matter of triage.  The bugs that must be
fixed are those that happen most often to the most users.

Traditionally, user feedback comes in the form of calls or e-mail to a
technical support center.  Technically savvy users might be able to
provide a stack trace, core dump, or even a repeatable test case.  But
for the most part, software designers have only an impoverished view
of how their code is operating (and failing) in the real world.

Widespread Internet connectivity has improved this somewhat.
Netscape/Mozilla, Microsoft, GNOME, and KDE have all deployed
automated (but opt-in) crash reporting systems.  These can report key
information about program state \emph{after} a failure has occurred:
stack trace, register contents, and the like.  However, program
actions \emph{before} the crash remain unknown.  Trace information
leading up to the failure may contain critical clues, especially in
such cases as heap corruption where the erroneous program action may
occur well before the resultant crash.  Also, such systems report no
information for successful runs.  This makes it difficult to
distinguish anomalous (crash-causing) behavior from behavior common to
all runs.

Therefore, we wish to trace or monitor program behavior continuously
throughout a run.  Furthermore, we want to measure reality: actual use
by actual users.  Any monitoring scheme, then, must be sufficiently
lightweight as to not interfere with normal program usage.  The size
of collected data is a concern as well, in terms of both the storage
space and the network bandwidth required to collect it.  Working to
our advantage is the sheer number of runs we may examine: any deployed
software system will have far more end users than dedicated in-house
testers.

Because of this advantage of numbers, it is not necessary to gather
complete information about each and every run.  We can collect just a
small sample of program behavior on each run.  The aggregate picture
resulting from a large number of such samples will still be a truthful
representation of overall program usage, provided that the samples are
collected in a statistically fair, randomized manner.  Sampling can
address performance and size concerns as well: when the monitoring
cost is spread thin across a great many users, the incremental burden
on any single user can be made arbitrarily small.

This paper presents a framework for fair random sampling of program
behavior.  Section~\ref{sec:framework} presents the program
transformation at the heart this framework, which allows us to add
randomly sampled instrumentation code to an existing application.  In
this section we also discuss some of the design issues
\unknown[something about keeping data size down, and doing client-side
data reduction].  Section~\ref{sec:applications} presents two example
applications of our system: sharing the cost of checking assertions,
and hunting for bugs using trace data mining.  Monitoring of deployed
code raises important privacy and security concerns, both social and
technical; a discussion of these issues appears in
Section~\ref{sec:privsec}.  In Section~\ref{sec:related-work} we
discuss related work, and Section~\ref{sec:conclusions} concludes.

\aside{Did this introduction take too long to get to the point?}

\aside{I fear that my message may be a bit muddled.  I just care about
  fixing bugs, or do I care about building up a broader understanding
  of what the code is doing?}
