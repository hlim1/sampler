\section{Privacy and Security}
\label{sec:privsec}

As noted in our introduction, the most important program behaviors are
those exhibited by deployed software in the hands of real end users.
However, any scheme for monitoring software post-deployment
necessarily raises important privacy and security concerns.  The
issues are complex, and as much social as technical.  However, our
approach will only succeed to the extent that users feel safe
contributing to the shared data pool, so these concerns must be
addressed.

\begin{figure}
  \centering
  \newlength{\boxwidth}
  \setlength{\boxwidth}{\columnwidth}
  \addtolength{\boxwidth}{-\leftmargin}
  \addtolength{\boxwidth}{-\rightmargin}
  \fbox{\parbox{\boxwidth}{\small
      
      \addtolength{\parskip}{1ex}

      The Netscape Quality Feedback Agent is a feature that gathers
      predefined technical information about Communicator and sends it
      back to the Netscape software development team so they can
      improve future versions of Communicator.
      
      \dots
      
      No information is sent until you can examine exactly what is
      being sent.
  
      \dots
  
      Information gathered by this agent is limited to information
      about the state of Communicator when it has an error. Other
      sensitive information such as web sites visited, email messages,
      email addresses, passwords, and profiles will not be collected.
  
      All information Netscape collects via this agent will be used
      only for the purposes of fixing product defects and improving
      the quality of Netscape Communicator. This data is for internal
      diagnostic purposes only and will not be shared with third
      parties.
  
      For more information on Netscape's general privacy policy, go
      to: \url{http://home.netscape.com/legal_notices/privacy.html}
  
      Communicator activates the agent dialog box when a problem
      occurs, or when it has gathered information that Netscape needs
      to improve future versions of Communicator.
    
      \dots
  
      If you prefer to disable the agent, you may do so here:}}
  
  \caption{Privacy assurances as used in Netscape Quality Feedback Agent}
  \label{fig:netscape}
\end{figure}

The experiences of Netscape/Mozilla with crash feedback systems may be
illustrative.  We have met with members of the Netscape Talkback Team,
a group of quality assurance engineers specifically tasked with
managing crash reports from the automated feedback system.
Considerable effort has gone into designing the client side of this
system so that users are fully informed.  The system is strictly
opt-in on a per-failure basis, or may be disabled entirely.  The user
may optionally examine the contents of the crash report, and no
information is ever sent to Netscape without explicit authorization
from the user.  Figure~\ref{fig:netscape} shows the sort of
information presented each time Netscape or Mozilla has crash data to
submit.

Of course, not all users will read or understand these disclaimers.
Furthermore, while one can visually scan a Mozilla crash report for
incriminating strings, users cannot be expected to make informed
privacy judgments about an invariants-checking report that reveals
that \texttt{indx} was found to be greater than \texttt{a_count} 5280
times.

Even so, there are some technical measures we can take to protect the
privacy of even non-technically savvy users.  The very nature of the
sampling process itself affords a degree of anonymity.  We collect a
small bit of information from many, many users; any single run has
little revelatory power, and the information which is revealed will
generally be so randomly scattered as to make coherent reassembly of a
compound value, such as a password string, infeasible.

\aside{Statistics facet of things can yield anonymity: keep subtotals
  only, never store original data.}

\aside{Non-interference, taint analysis, etc. might help identify data
  which should never be leaked into a sample.}

\aside{Briefly mention database poisoning by malicious or selfish
  clients}

