\section{Sampling Framework}
\label{sec:framework}

This section describes our sampling framework.  We begin
with sampling of basic blocks and gradually add features
until we can describe how to perform sampling for entire
programs.  Suppose we start with the following C code:

\begin{code}
  \{\+ \\
  check(p != 0); \\
  \up p = p->next; \\
  \\
  check(i < max); \\
  \up total += sizes[i]; \\
  \<\}
\end{code}

Our sampling framework can be configured to sample arbitrary pieces of
code, which may be either portions of the actual program or
instrumentation predicates added to the program.  For this particular
example, assume that the italicized \texttt{\textit{check()}} calls
have been tagged for sampling.  A \texttt{\textit{check()}} call might
conditionally halt the program (as \texttt{assert()}), or it might
append an event to a trace stream, or it might update a per-predicate
counter to record how often the predicate is true.  The precise
behavior of the instrumentation code is of no concern to the sampling
transformation itself.

\subsection{Sampling the Bernoulli Way}

Suppose that we wish to sample one hundredth of these checks.
Maintaining a global counter modulo one hundred is simple, but has the
disadvantage of being trivially periodic.  If the above fragment were
in a loop, for example, one of the checks would execute on every
fiftieth iteration while the other would never execute.  We wish
sampling to be fair and random: each check should independently have a
1/100 chance of being sampled.  This is a so-called \termdef{Bernoulli
process}, the most common example of which is repeatedly tossing a
coin.  We wish to sample based on the outcome of tossing a coin which
is biased to come up heads only one time in a hundred.

A \naive approach would be to use a simple random number
generator.  Suppose \texttt{rnd($n$)} yields
a random integer uniformly distributed between 0 and $n-1$.  Then the
following code gives us fair random sampling at the desired
density:

\begin{code}
  \{\+ \\
  if(rnd(100) == 0) check(p != 0); \\
  \up p = p->next; \\
  \\
  if(rnd(100) == 0) check(i < max); \\
  \up total += sizes[i]; \\
  \<\}
\end{code}

This strategy has some practical problems.  Random number generation
is not free: tossing the coin may be slower than simply doing the
check unconditionally.  Furthermore, what was previously straight-line
code is now dense with branches and joins, which may impede other
optimizations.

Sampling is sparse.  Each of the conditionals has a 99/100 = 99\%
chance of not sampling.  On any run through this block, there is a
$(99/100)^2 \approx 98\%$ chance that both instrumentation sites will
be skipped.  If we could determine, upon reaching the top of a basic
block, that no site in that block will be sampled, then we could
branch off into fast-path code from which all conditionally-guarded
checks have been removed.  This requires two versions of the code:
one with sampled instrumentation, one without.  It also requires that
we be able to predict how many future sampling opportunities will be
skipped before the next one is taken.

Anticipating future samples requires a change in randomization
strategy.  Consider a sequence of biased coin tosses, with ``0''
indicating no sample and ``1'' indicating that a sample is to be
taken.  Temporarily increasing the sampling density to 1/5, we might
have:
%%
\begin{equation*}
  \langle
    \underbrace{0, 0, 0, 0, 0, 1,}_{\text{6 tosses}}
    \underbrace{0, 0, 0, 1,}_{\text{4 tosses}}
    \underbrace{0, 1,}_{\text{2 tosses}}
    \underbrace{0, 0, 1}_{\text{3 tosses}}
    \rangle
\end{equation*}

A more compressed representation would be to encode the number of
tosses until (and including) the next sampled check: $\langle 6, 4, 2,
3 \rangle$.  An advantage of this representation is that it is
predictive: looking at the head of the sequence, we can see that five
sites will be skipped before the next sample.  The head of the
sequence can be treated as a countdown, telling us how far away the
next sample is.  If we are at the top of a basic block containing only
two checks, and the next sampling countdown is 6, we know in advance
that neither of those sites will be sampled on this visit.  Instead,
we merely discard two tosses and proceed directly to our
instrumentation-free fast path:

\begin{code}
  \{\+ \\
  if(countdown > 2) \{ \\
  \> /* fast path: no sample in near future */ \\
  \> countdown -= 2; \\
  \> \up p = p->next; \\
  \> \up total += sizes[i]; \\
  \} else \{ \\
  \> /* slow path: sample is imminent */ \\
  \> if(countdown-- == 0) \{ \\
  \>\> check(p != 0); \\
  \>\> countdown = getNextCountdown(); \\
  \> \} \\
  \> \up p = p->next; \\
  \> \\
  \> if(countdown-- == 0) \{ \\
  \>\> check(i < max); \\
  \>\> countdown = getNextCountdown(); \\
  \> \} \\
  \> \up total += sizes[i]; \\
  \} \\
  \<\}
\end{code}

The instrumented code must do some extra work to manage the
next-sample countdown, but the fast path is much improved.  The only
overhead is a single compare/branch and a constant decrement, and this
overhead is amortized over the entire block.  In larger blocks with
more instrumentation sites, the initial countdown check has a
larger threshold, but that one check suffices to predict a larger
number of skipped sampling opportunities.

Consider the distribution of likely countdown values.  With a sampling
density of 1/100, there is a 1/100 chance that we sample at the
very next opportunity.  There is a $(99/100) \times (1/100)$ that the
next chance is skipped but that the one after that is taken.
A countdown of three appears on a $(99/100)^2 \times (1/100)$ chance,
and so on.  These numbers form a \termdef{geometric distribution}
whose mean value is the inverse of the sampling density (that is,
100).  Numbers in a geometric distribution characterize the expected
inter-arrival times of a Bernoulli process.  However, repeated tossing
of a biased coin is not necessary: geometrically distributed random
numbers can be generated directly using a standard (uniform) random
generator and some simple floating-point operations.\footnote{In
  theory, a countdown may need to be arbitrarily large.  However, the
  odds of a 1/100 countdown exceeding $2^{32}-1$ are less than one in
  $10^{10^7}$.  This divisor vastly exceeds any reasonable estimate of
  the number of elementary particles in the universe.}

As can be seen in the instrumented slow path, the countdown must be
reset once it reaches zero.  Thus, we consume next-sample countdowns
gradually over time.  However, the rate of consumption is slower
than that for raw coin tosses: $n$ countdowns for $1/d$ sampling
encode, on average, $nd$ tosses.  A bank of pre-generated random
countdowns, then, is quite reasonable and will exhaust or repeat $d$
times more slowly than would a similar bank of raw coin tosses.

\subsection{From Blocks to Functions}

The block amortizing scheme outlined generalizes to an
arbitrary control flow graph as follows.  Any region of
acyclic code has a finite number of possible paths.
We call the maximum number of instrumented sites on any path the
\termdef{weight} of that region.  If the next-sample countdown exceeds
the weight of an acyclic region, then no samples will be taken on this
pass through that part of the code.  We can place a countdown
threshold check at the top of this region.

Cycles that without instrumentation are effectively weightless and may
be disregarded.  A cycle with instrumentation must also contain a
threshold check.  This guarantees that if we start at any threshold
check and execute forward, we cross a finite number of instrumentation
sites before reaching the next threshold check.  Thus, we can compute
a finite weight for each threshold check.

There is some flexibility regarding exactly where
a threshold check is placed.  For simplicity, our present system
places one threshold check at function entry and one along each loop back
edge.  Weights may be computed in a single bottom-up traversal of each
function's control flow graph.  If any threshold check is found to
have zero weight, it is simply discarded.

We produce two complete copies of the function body.  One contains
full instrumentation, with each possible sample guarded by a decrement
and test of the next-sample countdown.  The other copy, the fast path,
merely decrements the countdown where appropriate, but otherwise has
all instrumentation removed.  We stitch the two variants together at
threshold check points: at the top of each acyclic region, we decide
whether a sample is imminent.  If it is, we branch into the
instrumented code.  If the next sample is far off, we continue in the
fast path code instead.


\begin{figure}
  \centering
  %% -*- LaTeX -*-

\CompileMatrices
\xymatrix@=10pt@d{
  &
  &
  *++[o][F.]{} \ar[r] &
  *++[o][F]{} \ar[r] &
  *++[o][F]{} \ar[rr] \ar[dr] &
  &
  *++[o][F.]{} \ar[r] &
  *++[o][F]{} \ar[r] \ar@(l,d)[llldd] &
  *++[o][F.]{} \ar[ddr] \\
  %%
  &
  &
  &
  &
  &
  *++[o][F.]{} \ar[ur] \\
  %%
  \ar[r]
  &
  *++[F-]{>4?} \ar[uur] \ar[ddr] &
  &
  &
  *++[F-]{>3?} \ar@(u,l)[uul] \ar@(u,r)[ddl] &
  &
  &
  &
  &
  \\
  %%
  \\
  %%
  &
  &
  *++[o][F.]{} \ar[r] &
  *++[o][F]{} \ar[r] &
  *++[o][F]{} \ar[rr] \ar[dr] &
  &
  *++[o][F.]{} \ar[r] &
  *++[o][F]{} \ar[r] \ar@(r,d)[llluu] &
  *++[o][F.]{} \ar[uur] \\
  %%
  &
  &
  &
  &
  &
  *++[o][F.]{} \ar[ur] \\
}


%% LocalWords: ur uur ddr ddl llluu llldd rr dr uul

  \caption{Example of instrumented code layout}
  \label{fig:code-layout}
\end{figure}

\autoref{fig:code-layout} shows an example of code layout for a
function containing one conditional and one loop.  Dotted nodes
represent instrumentation sites; these are reduced to countdown
decrements in the fast path.  The boxed nodes represent threshold
checks; we have added one at function entry and one along the back
edge of the loop.  This code layout strategy is a
variation on that used by Arnold and Ryder to reduce the cost of code
instrumented for performance profiling \cite{Arnold:2001:FRC}.  The
principal change in our transformation is the use of geometrically
distributed countdowns in conjunction with acyclic region weights to
choose between the two code variants.  Arnold and Ryder use fixed
sampling periods (e.g., exactly once per $n$ opportunities, or exactly
once per $n$ instructions) and do not apply region-specific weighting.
Our approach imposes more overhead, but offers greater statistical
rigor in the resultant sampled data.  Arnold and Ryder have studied
several variant layouts with varying trade-offs of code size versus
overhead; the same choices and trade-offs are directly applicable
here.

\subsection{Function Calls}

New choices arise in the presence of function calls.  A conservative
treatment assumes any function call changes the countdown arbitrarily.
Therefore, a new threshold check must appear immediately after each
function call.  This treatment is appropriate if, e.g., the callee is
being compiled separately.

However, if the callee is known and available for examination, a
simple interprocedural analysis can be used.  A
\termdef{weightless function} has the following properties:

\begin{itemize}
\item The function contains no instrumentation sites.
\item The function only calls other weightless functions.
\end{itemize}

The set of weightless functions can be computed iteratively, requiring
no more steps in the worst case then the depth of the deepest
non-recursive call chain.

For purposes of identifying acyclic regions and placing threshold
checks, calls to weightless functions are invisible.  Acyclic regions
can extend below such calls, and no additional threshold check is
required after such a call returns.  A further benefit is that the
bodies of weightless functions may be compiled with no modifications.
They have no threshold checks, no instrumented code, and therefore
require no cloning or transformation of any kind.

\subsection{Global Countdown Management}

Our initial experience suggests that having the next-site countdown in
a global variable can be expensive.  Our system is implemented as a
source-to-source transformation for C, with \texttt{gcc} as our native
compiler.  Examining the native machine code reveals that \texttt{gcc}
treats the many ``\texttt{countdown--}'' decrements along the fast
path quite poorly.  It will not, for example, coalesce a sequence of
five such decrements into a single ``\texttt{countdown -= 5}''
adjustment.  This apparently stems from conservative assumptions about
aliasing of global variables.

Efficient countdown management requires that the native C compiler
take greater liberties when optimizing these decrements.  We assist
the native compiler by maintaining the countdown in a local variable
within each function:

\begin{enumerate}
\item At function entry, \termdef{import} the current global countdown
  into a local variable.
\item Use this local copy for all decrements, threshold checks, and
  sampling decisions.
\item Just before function exit, \termdef{export} this local copy back
  out to the global.
\end{enumerate}

In order to maintain agreement across all functions, we must also
export just before each function call and import again after each call
returns.  Again, though, calls to weightless functions may simply be
ignored.  They will not change or even inspect the countdown.
Similarly, the bodies of weightless functions need not import and
export at entry and exit, since they always leave the countdown
unchanged.  With this change, the conventional native C compiler can
coalesce decrements and perform other standard but important
optimizations.
