\section{Applications and Experiments}
\label{sec:applications}

We have used the transformation framework to inject different forms of
instrumentation into several programs, both to assess the performance
impact of sampling as well as to show that sampling can offer novel
insights into program (mis)behavior.  We report on our findings here.

\subsection{Sharing the Cost of Assertions}

In conventional usage, C \texttt{assert()} calls are used during
program development but are disabled (``\texttt{-DNDEBUG}'') when the
code ships in order to boost performance.  However, we all know that
shipping release 1.0 does not magically make our code bug-free.
Deployed programs will fail in unanticipated ways, and it would be
helpful to retain some level of assertion checking if the performance
penalty were not excessive.

As an extreme example of assertion-dense code, consider software built
using the \CCured instrumenting translator \cite{POPL_'02*128}.
\CCured analyses programs written in C and attempts to prove,
statically, that most pointer operations are memory safe.  Where this
cannot be done, \CCured inserts dynamic checks to enforce memory
safety at run time.  For purposes of our research, \CCured may be
thought of as a source of highly assertion-dense code.  The individual
assertions are quite small and fast: array bounds checks, testing for
null, etc.  However, in large numbers, their performance impact can be
considerable.  We wish to use random sampling to share this cost
fairly among many users.

We have applied sampling to several benchmarks taken from the
SPECINT95 \cite{SPEC95} and Olden \cite{Carlisle:1996:OPPWDDSDMM}
benchmark suites.  These are the same benchmarked originally examined
by \CCured's designers, but the memory safety bugs they discovered
have been fixed.  Thus all programs run to completion and we are
simply measuring the overhead of performing the dynamic checks.

\begin{table*}[tb]
  \centering
  \begin{tabular}{|l|rrr|rrr|}
    \hline
    & \multicolumn{3}{c|}{\textbf{function counts}} & \multicolumn{3}{c|}{\textbf{average for functions with sites}} \\
    \raisebox{1.5ex}[0pt]{\textbf{benchmark}} & \textbf{total} & \textbf{weightless} & \textbf{has sites} & \textbf{sites} & \textbf{threshold checks} & \textbf{threshold weight} \\
    \hline\hline
    bh & 66 & 14 & 50 & 11.8 & 3.7 & 9.4 \\
bisort & 15 & 3 & 11 & 3.5 & 1.8 & 2.3 \\
em3d & 17 & 5 & 11 & 5.3 & 2.9 & 4.6 \\
health & 18 & 2 & 15 & 5.5 & 2.7 & 3.0 \\
mst & 18 & 6 & 11 & 6.2 & 2.5 & 3.9 \\
perimeter & 11 & 4 & 6 & 6.3 & 2.7 & 1.8 \\
power & 22 & 5 & 17 & 4.9 & 2.7 & 2.7 \\
treeadd & 9 & 2 & 6 & 2.8 & 1.8 & 2.2 \\
tsp & 14 & 5 & 8 & 14.1 & 3.9 & 3.5 \\

    \hline
    compress & 23 & 4 & 18 & 6.3 & 2.6 & 3.7 \\
go & 382 & 12 & 361 & 14.7 & 5.9 & 4.7 \\
ijpeg & 316 & 27 & 272 & 18.5 & 5.0 & 7.3 \\
li & 377 & 16 & 336 & 6.2 & 3.2 & 2.9 \\

    \hline
  \end{tabular}
  \caption{Static metrics for \CCured benchmarks.  Olden benchmarks
    are listed first, followed by SPECINT95.}
  \label{tab:ccured-static}
\end{table*}

\autoref{tab:ccured-static} summarizes static aspects of the sampling
transformation.  For each benchmark, we give the total number of
non-library functions and the number of these which are weightless.
\CCured is a whole-program analysis, so weightless function
identification has the advantage of being able to examine every
function body.  We also count the number of functions which directly
contain at least one instrumentation site.  (What remains are those
functions which contain no sites of their own but which call other
functions that do have sites.)

Considering just the functions which directly contain at least one
instrumentation site, \autoref{tab:ccured-static} also presents the
average number of sites per function, the average number of threshold
check points per function, and the average threshold weight for all
such points.  (Note that the product of the last two of these metrics
may exceed the first, as a single instrumentation site may fall under
more than one threshold check point.  This can be seen in the example
in \autoref{fig:code-layout} as well.)  The average site count gives
an impression of how assertion-dense the code is, while the average
threshold weight measures how effective our transformation has been in
amortizing the cost of countdown checks over multiple sites.

\begin{table}
  \centering
  \begin{tabular}{|l|rrrr|}
    \hline
    \rule{0pt}{2.5ex}
    \textbf{benchmark} & $\mathbf{10^{-2}}$ & $\mathbf{10^{-3}}$ & $\mathbf{10^{-4}}$ & $\mathbf{10^{-6}}$ \\
    \hline\hline
    bh & 1.06 & 1.14 & 1.15 & 1.15 \\
bisort & 1.04 & 1.06 & 1.07 & 1.06 \\
em3d & 1.15 & 1.20 & 1.21 & 1.21 \\
health & 1.00 & 1.02 & 1.02 & 1.02 \\
mst & 1.06 & 1.07 & 1.07 & 1.07 \\
perimeter & 0.96 & 0.97 & 0.97 & 0.97 \\
power & 0.97 & 1.00 & 1.00 & 1.00 \\
treeadd & 1.05 & 1.05 & 1.05 & 1.05 \\
tsp & 1.02 & 1.02 & 1.02 & 1.02 \\

    \hline
    compress & 1.32 & 1.38 & 1.39 & 1.39 \\
go & 0.83 & 0.97 & 1.00 & 1.00 \\
ijpeg & 1.24 & 1.32 & 1.34 & 1.34 \\
li & 1.07 & 1.16 & 1.16 & 1.19 \\

    \hline
  \end{tabular}
  \caption{Relative speedup of various sampling rates versus always checking}
  \label{tab:ccured-density}
\end{table}

\autoref{tab:ccured-density} shows the performance effect of various
sampling densities.  We take the running time of standard \CCured code
as the baseline.  We report the speedup (>1) or slowdown (<1) relative
to this baseline when sampling at various densities.  All benchmarks
were compiled using \texttt{gcc} 2.96 using standard optimization
(\texttt{-O2}).  Times were collected on a \unknown MHz Pentium III
Linux machine with one gigabyte of RAM.  Reported speedups represent
the average of four runs; each run used a different pre-generated bank
of 1024 geometrically distributed random countdowns.

Even at a fairly high sampling density of 1/100, more than two thirds
of our benchmarks run faster then they would if all checks were
performed.  Because each single check is so small and fast, this
suggests that we have largely been successful in amortizing our
sampling overhead.  We are, in a sense, hiding the sampling overhead
in the time that would otherwise have been taken to run 99/100 checks.
The most dramatic speedups are seen in \texttt{ijpeg} (24\% faster)
and \texttt{compress} (32\%) faster.  On the other hand, three of the
benchmarks do run slower, with \texttt{go} showing the largest
penalty.  In these cases, the time recovered by skipping 99/100 checks
is not enough to mask the added overhead of the sampling
infrastructure.

As we reduce the sampling density to 1/1,000, \texttt{power} reaches a
balance point of no measurable speedup or slowdown, while
\texttt{perimeter} remains a stubborn 3\% slower.  The \texttt{health}
benchmark crosses over and is now slightly faster with sampling.
Other benchmarks, which already showed speedups at 1/100, either
retain their position (\texttt{treeadd} and \texttt{tsp}) or show
additional boots, with \texttt{compress} reaching an overall 38\%
speedup versus standard \CCured.  Further reducing the sampling
density to 1/10,000 shows little change, and by the time we reach
1/1,000,000 it is clear that we have reached a performance ceiling.
One benchmark continues to run slower than standard \CCured; two run
at the same speed; the remaining ten run 2\% - 39\% faster.

\aside{I've helped the user interpret the table, but I don't really
  say anything deep about \emph{why} the numbers turn out the way they
  do.  Should I?  Do I actually know what to say by way of
  explanation?}

\aside{Might be nice to present speedup of code with all samples
  statically removed.  This represents the best we could ever possibly
  hope to do.}

\aside{Present and interpret single-function sampling data.}

\subsection{Finding Bugs Using Data Mining}
\label{sec:applications:mining}

Sampled information about program behavior can be thought of as a
database.  Data mining techniques, then, may reveal important
information about the aggregate behavior of many runs.  Of particular
interest to us is finding and fixing bugs.  We now illustrate how
mining of sampled program behavior can help direct software engineers
to the root cause of difficult bugs.

Because we do not have ready access to thousands of users, we simulate
a large user community by using randomly generated inputs in the
spirit of the Fuzz project~\cite{MKLMMNS95}.  Our selected target is
version 1.06 of the GNU implementation of \texttt{bc}.  We find that
feeding \texttt{bc} nine megabytes of random input causes it to crash
roughly one time in four.  While this would be an unusually high
failure rate for shipping software under normal usage patterns, we
expect that users experiencing crashes will be more likely to opt-in
to a trace reporting system.  Thus, this may be a reasonable ratio to
expect if most crashes and a few non-crashes are reported.

Quick perusal of the stack trace at a typical failure is discouraging.
The crash occurs several frames deep inside the C library's
\texttt{malloc()} function: a sure sign of heap corruption.  Such bugs
are especially pernicious because the actual corruption may have
happened well before the ultimate crash \cite{Eisenstadt1993b}.

We instrument \texttt{bc} to guess and randomly check a large number
of possible invariants.  Our goal is to identify invariants that hold
when the program succeeds, but which are violated when the program
crashes.  We cast an extremely broad net, but with an eye toward
pointer use violations and buffer overruns.  At any direct assignment
to a scalar variable $x$, we identify all other local or global
variables $\{ y_1, y_2, \dots, y_n \}$ which are also in scope and
which have the same type.  We then compare $x$ to each $y_i$, and
update one of three counters depending on whether $x$ was smaller,
equal to, or greater than $y_i$.  A distinct counter triple is
maintained for each $(x, y_i)$ pair at each distinct syntactic
assignment.  We compare pointers to same-typed pointers as well, and
additionally compare each pointer for equality with null.  One
comparison between $x$ and $y_i$, which bumps one of three counters,
is considered to be one instrumentation site subject to random
sampling.

As expected, this gives us a large number of candidate invariants:
13,442 counter triples, or 40,326 counters in all.  Because our
invariant guesses are so crude, the vast majority of these are of no
interest, either because they compare completely unrelated variables
or because they express invariants which behave indistinguishably in
both successful and failed runs.

To find the few invariants that do count, we use \aside{identify and
  describe the particular form of logistic regression we used.}

\aside{Interpretation of logistic regression results.  Statement that
  they direct us to some small region of code in a single function,
  versus 9764 lines of code in all of \texttt{bc}.  Description of
  previously unreported buffer overrun bug found on like 176 of
  \texttt{storage.c}.
  
  Yes, this is rigged.}

\aside{Alice reports:
  
  Well, of the 40296 features (yet another name for ``predictors''),
  only 9056 of them are ever non-zero.  Of the non-zero useful
  features, 955 are constant; 948 of these are constant at 1, 2 at 2,
  1 at 28, 1 at 31, and 3 at 32.  144 of the useful features have very
  small variation, where the value is only different for one or two
  runs.
  
  Beyond what this does for the regression analysis, this means the
  counts would compress very well.  That's good news for deployed
  users with slow modems.  It's also good news for the server's hard
  drive.}

\aside{While it is uncommon to do an analysis where the number of
  independent variables swamps the number of observations, there is
  some precedent, especially in biology.  For example:

  \begin{itemize}
  \item 81 samples in 3 classes; 4,682 genes \cite{Dudoit:2002:CDM}
  \item 72 samples in 3 classes; 6,817 genes \cite{Dudoit:2002:CDM}
  \item 64 samples in 9 classes; 5,244 genes \cite{Dudoit:2002:CDM}
  \item 1909 cases (42 positive) in 2 classes; 139,351 binary features
    \cite{2002-cheng}
  \end{itemize}
  
  Although the specific techniques vary, these folks (like us) use
  methodologies which assume that the vast majority of predictors are
  actually irrelevant, and you want to pick out the few that actually
  matter.}

