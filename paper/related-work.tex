\section{Related Work}
\label{sec:related-work}

Sampling has a long history, with most applications focusing on
performance profiling and optimization.  Any sampling system must
define a trigger mechanism that signals when a sample is to be taken.
Typical triggers include periodic hardware timers/interrupts
\cite{Burrows:2000:EFV,Traub:200:EILPP,Whaley:337483}, periodic
software event counters (e.g.  every $n$th function call)
\cite{Arnold:2000:ACCTS}, or both.  In most cases, the sampling
interval is strictly periodic; this may suffice when hunting for large
performance bottlenecks, but may systematically miss rare events such
as crash-causing invariant violations.

The Digital Continuous Profiling Infrastructure
\cite{Anderson:1997:CPW} is unusual in that sampling intervals are
chosen randomly.  However, the random distribution is uniform, such as
one sample every 60K to 64K cycles.  Samples thus extracted are not
statistically fair in the sense of a Bernoulli (coin-tossing) process.
If one sample is taken, there is zero chance of taking any sample in
the next 1-59,999 cycles.  Similarly, there is zero chance of
\emph{not} taking exactly one sample in the next 60K-60K cycles.  We
trigger samples based on a geometric distribution, which correctly
models the time between independently randomized coin tosses.  The
resulting data is a statistically rigorous fair random sample, which
in turn grants access to a large domain of powerful statistical
analysis methodologies.

Recent work in trace collection is broadening in scope from pure
performance to more general program understanding.  Techniques for
capturing program traces confront challenges similar to those we face
here, such as minimizing performance overhead and managing large
quantities of captured data.  Dynamic analysis in particular must
encode, compress, or otherwise reduce an incoming trace stream in real
time, as the program runs \cite{Demsky:RBEOOP:2002,ICSE01*221}.  It
may be difficult to directly adapt dynamic trace analysis techniques
to a domain where the trace is sampled and therefore incomplete.
However, it may be possible to combine these approaches so that one
fairly and randomly samples short trace ``bursts'', with complete data
collection within each burst.  This would allow us to examine
sequences of interrelated events, a task which is difficult under our
current, strictly Bernoullian regime.

Our effort to understand and debug programs by identifying invariants
relates to prior work by the Daikon project \cite{ernst2001}.  Like
Daikon, we begin with fairly unstructured guesses as to possible
invariants, and eliminate those which do not appear to hold.  Daikon
collects a complete trace of individual program values; generation and
testing of possible invariants occurs off line.  This gives Daikon
greater flexibility to consider a much larger family of possible
invariants then we have examined here, as a significant part of our
invariant testing happens within the client itself.  With a large
enough user population, though, the myriad invariants considered by
Daikon could certainly be farmed out and tested in the field.

The DIDUCE project \cite{Hangal:DIDUCE:2002} also attempts to identify
program invariants.  Unlike Daikon, most processing does take place
within the client program.  As in our study, DIDUCE attempts to relate
changes in invariants to the manifestation of bugs.  However, DIDUCE
performs complete tracing and focuses on discrete state changes, such
as the first time an invariant was seen to be false.  Our approach is
more probabilistic: we with to identify broad trends over time that
correlate invariant violations with increased likelihood of failure.
Our approach explicitly accommodates the possibility of ``getting
lucky'', such as overrunning a buffer but not actually crashing.
DIDUCE works in the context of a safe language (Java) in which most
forms of getting lucky cannot happen.
