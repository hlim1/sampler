\section{Related Work}
\label{sec:related-work}

Sampling has a long history, with most applications focusing on
performance profiling and optimization.  Any sampling system must
define a trigger mechanism that signals when a sample is to be taken.
Typical triggers include periodic hardware timers/interrupts
\cite{Burrows:2000:EFV,Traub:200:EILPP,Whaley:337483}, periodic
software event counters (e.g., every $n$th function call)
\cite{Arnold:2000:ACCTS}, or both.  In most cases, the sampling
interval is strictly periodic; this may suffice when hunting for large
performance bottlenecks, but may systematically miss rare events.

The Digital Continuous Profiling Infrastructure
\cite{Anderson:1997:CPW} is unusual in choosing sampling intervals
randomly.  However, the random distribution is uniform, such as one
sample every 60K to 64K cycles.  Samples thus extracted are not
independent.  If one sample is taken, there is zero chance of taking
any sample in the next 1--59,999 cycles and zero chance of \emph{not}
taking exactly one sample in the next 60K--64K cycles.  We trigger
samples based on a geometric distribution, which correctly models the
interval between successful independent coin tosses.  The resulting data is a
statistically rigorous fair random sample, which in turn grants access
to a large domain of powerful statistical analyses.

Recent work in trace collection has focused on program understanding.
Techniques for capturing program traces confront challenges similar to
those we face here, such as minimizing performance overhead and
managing large quantities of captured data.  Dynamic analysis in
particular must encode, compress, or otherwise reduce an incoming
trace stream in real time, as the program runs
\cite{Demsky:RBEOOP:2002,ICSE01*221}.  It may be difficult to directly
adapt dynamic trace analysis techniques to a domain where the trace is
sampled and therefore incomplete.  

Our effort to understand and debug programs by selecting predicates is
partly inspired by Daikon \cite{ernst2001}.  Like Daikon, we begin
with fairly unstructured guesses and eliminate those that do not
appear to hold.  Unlike Daikon, we are concerned with gathering
data from production code, which leads us to use sampling of a large
number of runs and statistical models; the Daikon experiments are
done on a smaller number of complete traces.  We are also interested
in detecting bugs, while Daikon focuses on the somewhat different problem of
detecting program invariants.

The DIDUCE project \cite{Hangal:DIDUCE:2002} also attempts to identify
bugs using analysis of executions.  Unlike Daikon, most
processing does take place within the client program.  As in our
study, DIDUCE attempts to relate changes in predicates to the
manifestation of bugs.  However, DIDUCE performs complete tracing and
focuses on discrete state changes, such as the first time a predicate
transitioned from true to false.  Our approach is more probabilistic:
we wish to identify broad trends over time that correlate predicate
violations with increased likelihood of failure.

\termdef{Software tomography} as realized through the GAMMA system
\cite{PASTE'02*2} shares our goal of low-overhead distributed
monitoring of deployed code.  Applications to date have focused on
code coverage and traditional performance monitoring tasks, whereas
our primary interest is bug isolation.  
%Our strategy uses
%randomization within a single instrumented binary, while GAMMA
%emphasizes choices in initial probe placement and iterative refinement
%over time.  Our earlier discussion of statically selective sampling
%(Section~\ref{sec:ccured:single}) suggests that these two
%considerations are complementary.

%% LocalWords: DIDUCE Whaley ICSE
